\chapter{{\tac}のアーキテクチャ}

{\tec}のアーキテクチャは「TeC教科書」\footnote{
\url{https://github.com/tctsigemura/TecTextBook/raw/master/tec.pdf}}
で詳しく説明されているので,
ここでは{\tac}のアーキテクチャについて簡単に説明する.

%-----------------------------------------
\section{CPUの概要}
{\tac}で使用できるデータの形式,
アドレス空間,
実行モード,
CPU内部のレジスタ構成,
機械語命令,割込みと例外について説明する.

\subsection{データ形式}
\figref{tacData}の「データ形式」に{\tac}が扱うことができるデータを示す.
16ビットの整数データと,16ビットのアドレスデータの他に,
8ビットの整数データを扱うことができる.
16ビットのデータはCPUの内部でもメモリやI/Oでも使用できる.
8ビットデータはメモリ上の配列に格納することだけを想定している.

\subsection{アドレス空間}
\figref{tacData}の「メモリ空間」,「I/O空間」に
{\tac}のメモリとI/Oのアドレス空間を示す.
メモリ空間もI/O空間もバイト単位でアドレスが割り付けてある.
しかし,I/O空間をバイト単位でアクセスする機械語命令を現在のTaCは備えていない
\footnote{以前のTaCはI/O空間もバイト単位でアクセスできたので,
バイト単位でアドレスが割り付けられている.}.
メモリやI/Oの16ビットデータにアクセスする場合は偶数番地を用いる.
8ビットデータはメモリの読み書きだけに使用できる.
メモリの8ビットデータにアクセスする場合は,
CPUレジスタの下位8ビットだけが使用される.

\subsection{実行モード}
{\tac}は「特権モード」,「ユーザモード」,「I/O特権モード」の
三つの実行モードを持っている.

\begin{description}
\item[特権モード]
全ての機械語命令が実行できるモードである.
OSカーネルは特権モードで実行される.
\item[ユーザモード]
実行モードを変更したり,
ハードウェアの状態を変更したりする\emph{特権命令}を実行することができない.
通常,ユーザプログラムはユーザモードで実行される.
\item[I/O特権モード]
IN,OUT機械語命令が実行できるユーザモードである.
入出力ポートに接続したオプションのハードウェア\footnote{
このようなハードウェアはOSによってサポート・管理されない.
}を使用するプリケーションを実行するために用意されている.
\end{description}

\subsection{CPUレジスタとPSW}
\figref{tacData}の「レジスタ構成」にCPU内部のレジスタなどを示す.
レジスタはどれも16ビット幅である.

\subsubsection{CPUレジスタ}
CPUレジスタは,
汎用のG0(General register 0)からG11,
フレームポインタとして使用するFP(Frame Pointer),
特権モード用のスタックポインタSSP(System Stack Pointer),
ユーザモード(I/O特権モードも含む)用の
スタックポインタUSP(User Stack Pointer)からなる.
これらは全て計算用にもアドレス用にも使用できる.
FP,SSP,USPは,以下に説明する特別な意味も持っている.

\subsubsection{フレームポインタ(Frame Pointer)}
フレームポインタ(FP)はCPUレジスタの一つである.
フレームポインタ相対アドレッシングモードで使用できる.
このアドレッシングモードを用いると,
スタックフレーム内のローカル変数や関数引数へ,
1ワード(2バイト)の機械語命令でアクセスできる.

\subsubsection{スタックポインタ(Stack Pointer)}
スタックポインタ(SP)もCPUレジスタの一つである.
{\tac}は特権モード用(SSP),
ユーザモード(I/O特権モード含む)用(USP)の
二本のスタックポインタを持っている.
SSPは特権モードでSPの位置にマップされ,
OSカーネル用のスタックポインタとして使用される.
USPはユーザモード(I/O特権モード含む)でSPの位置にマップされ,
ユーザプログラムのスタックポインタとして使用される.
USPはG14として常時マップされており,
特権モードでもUSPをアクセスすることができる.

\subsubsection{PSW(Program Status Word)}
PSWはPC(Program Counter)とFLAGからなる.
FLAGには,計算結果で変化する
\texttt{V(oVerflow)},
\texttt{C(Carry)},
\texttt{S(Sign)},
\texttt{Z(Zero)}と,
割込み許可\texttt{E(Enable interrupt)},
特権モード\texttt{P(Privilege)},
I/O特権モード\texttt{I(I/O Privilege)},
ユーザ定義\texttt{U(User defined)}の各ビットがある.

FLAGはG15として普通の機械語命令で扱うことも可能であるが,
ユーザモードでは\texttt{E},\texttt{P},\texttt{I}の各ビットは変化しない.

\subsection{機械語命令}
\figref{tacInst}に{\tac}の機械語命令の一覧表を示す.
HALTは特権モードでしか使用できない\emph{特権命令}である.
IN,OUTは特権モードとI/O特権モードで使用できる命令である.
これらの命令を非特権モードで実行すると特権違反例外が発生する.
SVC命令はシステムコールを発行するためにSVC例外を発生する.

ほとんどの転送命令と計算命令で8種類のアドレッシング・モードが使用できる.
Direct,Indexed,Immediateの
三つのアドレッシング・モードを使用する場合は2ワードの機械語命令になる.
他のアドレッシング・モードの場合は1ワード命令である.

Byte Register Indirect アドレッシング・モードだけが,
メモリの8ビットデータをアクセスする.
Byte Register Indirect アドレッシング・モードの
ST命令は,CPUレジスタの下位8ビットをメモリに書き込む.
これら以外の命令は,メモリから読み出した8ビットデータの上位に
\|00h|を付加した16ビットデータを使用する.

\subsection{割込み(Interrupt)・例外(Exception)}
{\tac}はベクタ方式(ベクタは\|FFE0h|番地〜)の割込み・例外機構を備えている.
割込み・例外の種類は\tabref{inter}に示す16種類である.
ゼロ除算や特権違反のような
ソフトウェアに起因する割込みを「例外(Exception)」と呼ぶ.
「TLBミス」から「SVC」までの6種類が「例外」 それ以外が「割込み」である.
「割込み」の許可と禁止はFLAGの\texttt{E}ビットを操作することで行う.
「例外」の発生は禁止できない.

割込みが発生すると次の順で割込み処理が行われる.
\begin{enumerate}
\item CPU内部の一時レジスタに\emph{FLAGのコピー}が作られる.
\item FLAGが変更され,
割込みが禁止(\texttt{E=0})の特権モード(\texttt{P=1})になる.
\item PCと\emph{FLAGのコピー}が順にカーネルスタックにPUSHされる.
\item PCに割込み(例外)ハンドラの開始番地がロードされ,
ハンドラの実行が開始される.
\end{enumerate}

\begin{mytable}{btp}{割込みの種類と意味}{inter}
  \begin{tabular}{ r  l | l }\hline\hline
    \multicolumn{2}{c}{割込み・例外} &
    \multicolumn{1}{|c}{意 味} \\\hline
    0 & Timer0      & ハードウェアタイマー0に設定された時刻になった.\\
    1 & Timer1      & ハードウェアタイマー1に設定された時刻になった.\\
    2 & RN4020受信  & Bluetoothモジュールから1バイトのデータを受信した.\\
    3 & RN4020送信  & Bluetoothモジュールへ1バイトのデータを送信し終えた. \\
    4 & FT232RL受信 & USBシリアル変換ICから1バイトのデータを受信した.\\
    5 & FT232RL送信 & USBシリアル変換ICへ1バイトのデータを送信し終えた. \\
    6 & TeC受信     & TeCから1バイトのデータを受信した. \\
    6 & TeC送信     & TeCへ1バイトのデータを送信し終えた. \\
    8 & マイクロSD  & マイクロSDのホストコントローラが
                      コマンドを実行し終えた.\\
    9 & PIO         & 入出力ポートの監視中のビットに変化があった. \\
    10& TLB ミス    & MMU有効時にTLBに必要なエントリが見つからない. \\
    11& メモリ保護違反 & 奇数アドレスでワードデータをアクセスした.または,\\
      &                & ページの保護モード(RWX)に
                         一致しないアクセスがされた. \\
    12& ゼロ除算    & 割り算機械語命令で「÷ 0」が実行された. \\
    13& 特権違反    & 不適切な実行モードで特権命令が実行された. \\
    14& 未定義命令  & {\tac}の機械語として解釈できない命令を実行した. \\
    15& SVC         & SVC 機械語命令が実行された. \\
  \end{tabular}
\end{mytable}

%-----------------------------------------
\section{メモリマップとI/Oマップ}
メモリ空間とI/O空間は8ビット毎にアドレス付けされている.
メモリは8ビットデータ,16ビットデータのどちらも読み書きできる.
I/Oは16ビットデータの読み書きしかできない.
メモリの場合は機械語命令のアドレッシング・モードによって,
8ビットデータと16ビットデータの区別をする.
16ビットデータは偶数アドレスを指定してアクセスしなければならない.

\subsection{メモリ空間}
\figref{tacMap}の「メモリマップ」に{\tac}のメモリマップを示す.
{\tac}のメモリ空間は\|0000h|から\|FFFFh|の64KiBである.
16ビットデータは偶数アドレスからの2バイトに配置され,
偶数アドレスを指定してアクセスする.
8ビットデータにアクセスするには,
Byte Register Indirect モードを用いる.
その他のアドレッシング・モードは,
16ビットデータをアクセスするために用いる.

リセット時に,\|E000h|から\|FFFFh|にIPL(ROM)が配置される.
{\tac}モードでは,IPLはマイクロSDからOSを読み出して起動する.
その他のモードでは,IPLが{\tec}の通信を中継する等の機能を果たす.
IPLはOSを読みだしたらIPL(ROM)を切り離しメモリ空間全体をRAMにした後,
OSに制御を渡す.
IPL(ROM)が切り離された後,
\|FFE0h|から\|FFFFh|は割込みベクタ領域になる.
16種類の割込み・例外ハンドラの入口番地をOSがセットする.

\subsection{I/O空間}
\figref{tacMap}の「I/Oマップ」に{\tac}のI/Oマップを示す.
{\tac}のI/O空間は\|00h|から\|FFh|の128ワードである.
I/O空間のアドレス幅は8ビットだが,
IN,OUT機械語命令ではI/Oアドレスが16ビットで表現される.
I/Oアドレスの上位8ビットは\|00h|になるようにする.
上位8ビットが\|00h|以外になった場合の動作は保証されない.
メモリ空間と異なり16ビットデータの読み書きしかできない.

%-----------------------------------------
\section{MMU(Memory Management Unit)}
ページング方式のMMUが使用できる.
MMUが働くのはユーザモードで実行中だけである.
MMUの制御は
\figref{tacMap}の「I/Oマップ」に示すポートをIN,OUT機械語命令で操作して行う.

\begin{center}
  \small\begin{tabular}{| r | c | c || c | c |}\hline
    \multirow{2}{*}{番地}
    & \multicolumn{2}{|c||}{IN}
    & \multicolumn{2}{c|}{OUT}
    \\\cline{2-5}
         & 上位バイト & 下位バイト & 上位バイト & 下位バイト
    \\\hline\hline
    80h  & 00 & TLB[0]上位8bit & - & TLB[0]上位8bit \\\hline
    82h  & \multicolumn{2}{|c||}{TLB[0]下位16bit}
         & \multicolumn{2}{|c|}{TLB[0]下位16bit} \\\hline
    84h  & 00 & TLB[1]上位8bit & - & TLB[1]上位8bit \\\hline
    86h  & \multicolumn{2}{|c||}{TLB[1]下位16bit}
         & \multicolumn{2}{|c|}{TLB[1]下位16bit} \\\hline
    ...  & \multicolumn{2}{|c||}{...}
         & \multicolumn{2}{|c|}{...}   \\\hline
    9Ch  & 00 & TLB[7]上位8bit & - & TLB[7]上位8bit \\\hline
    9Eh  & \multicolumn{2}{|c||}{TLB[7]下位16bit}
         & \multicolumn{2}{|c|}{TLB[7]下位16bit} \\\hline
    A0h  &  00 & 00
         &  -  & IPL切離し \\\hline
    A2h  &  \multicolumn{2}{|c||}{違反アドレス}
         &  -  & MMU有効化 \\\hline
    A4h  &  00 & 違反原因
         &  -  & - \\\hline
    A6h  &  00 & ページ番号
         &  -  & - \\\hline
  \end{tabular}
\end{center}

\subsection{違反アドレス}
  メモリ保護違反が発生した時,原因となった論理アドレスが記録される.

\subsection{違反原因(\texttt{0000 00BV})}
  メモリ保護違反が発生した時,
  奇数アドレスを用いたワードアクセス(Bad Address)の場合
  \texttt{B}ビットが`1'になる.
  ページの保護モード違反(Memory Violation)の場合
  \texttt{V}ビットが`1'になる.
  これらのビットはCPUがIN命令でA4h番地を読むとクリアされる.

\subsection{ページ番号}
  ページ番号がTLBでヒットしなかった場合 TLB ミス例外が発生する.
  その際,例外の原因となったページ番号が記録される.

\subsection{IPL切離し(\texttt{0000 000I})}
  \texttt{I}に`1'を書き込むと,
  物理メモリ空間最後の8KiBに配置されたIPL(ROM)が切り離されRAMに置き換わる.
  これによりメモリ空間64KiB全てがRAMになる.
  通常,切り離しの操作はIPL自身がOSをロードした後で自動的に行う.
  IPL(ROM)がマップされている時でもコンソールからはRAMが見えるので,
  IPLを最後まで実行していない場合は注意が必要である.

\subsection{MMU有効化(\texttt{0000 000E})}
  \texttt{E}に`1'を書き込むとMMUが有効になる.
  MMUが有効になるとCPUの実行モードが「I/O特権モード」,「ユーザモード」の時,
  ページからフレームへの変換($p \to f$変換)が行われる.

\subsection{TLB(\texttt{TLB[0]〜TLB[7])}}
  TLBはページ番号で検索されフレーム番号を出力し$p \to f$変換を行う.
  8エントリのTLBをIN,OUT機械語命令で参照・操作することができる.
  8つのTLBエントリはI/O空間の80h番地から9Eh番地までの範囲に配置される.
  1つのTLBエントリは24bitであるので,
  上位8bitと下位16bitに分けてアクセスする.
  以下ではTLBエントリの各ビット(フィールド)の意味を説明する.

  \begin{center}
  \small\begin{tabular}{| c || c |c|c|c|c|c|c| c |}\hline
    \multicolumn{9}{|c|}{TLBエントリの構成} \\\hline
    I/Oポート   &上位8bit&\multicolumn{7}{|c|}{下位16bit} \\\hline
    ビット番号  &23 - 16&15&14&13&12&11&10-8&7 - 0      \\\hline
    ビットの意味& ページ番号&
    \|V|&\|U|&\|U|&\|R|&\|D|&\|RWX|&フレーム番号\\\hline
  \end{tabular}
  \end{center}

\subsubsection{ページ番号(\texttt{p})} 
エントリが管理するページの番号を設定する.
TLBエントリをページ番号で検索し,
該当ページを管理するエントリが見つからない場合,TLB ミス例外が発生する.

\subsubsection{\texttt{V}(Valid)}
エントリが有効かどうかを表す.
TLBを検索する際,$V=0$のエントリーは無視される.

\subsubsection{\texttt{U}(Undefined)}
ハードウェアでは使用してないビットである.
単なる1ビットの記憶装置としてOSが自由に使用できる.

\subsubsection{\texttt{R}(Reference)}
OSがクリアしたあとユーザプログラムがページを参照すると`1'になる.
OSがページの参照に関する統計情報を取得するために使用する.

\subsubsection{\texttt{D}(Dirty)}
OSがクリアしたあとユーザプログラムがページに書き込みを行うと`1'になる.
OSがページをスワップアウトする必要があるか判断するために使用する.

\subsubsection{\texttt{RWX}(Read/Write/eXecute)}
ページの保護モードを3ビットの組み合わせで表現する.
機械語セグメントには\texttt{101},
データセグメントやスタックセグメントには\texttt{110},
読み出し専用のデータセグメントには\texttt{100}をセットする.
CPUが保護モードで許可されない種類のアクセスを行った場合,
メモリ保護違反例外が発生する.

\subsubsection{フレーム番号(\texttt{f})}
ページ番号に対応するフレーム番号を設定する.
MMUはページ番号がヒットしたエントリのフレーム番号を変換結果として出力する.

%-----------------------------------------
\section{IPLプログラム}
\label{ipl}
{\tac}はリセットされると自動的にIPLプログラム\footnote{
IPLのソースコードは
\url{https://github.com/tctsigemura/TeC7/tree/master/TaC/Ipl}
に公開されている.
}の実行を開始する.
IPLの第一の役割は,マイクロSDからOSを読出し起動することである.
しかし,{\tecS}の動作モード(\ref{tec7mode}参照)によっては,
{\tec}の補助(\ref{tec7assist}参照)を行う.
以下では動作モード毎にIPLの役割を説明する.

\subsection{{\tec}モード}
USBシリアル変換IC(FT232RL)から受信したデータを{\tec}のSIOへ送信する.
また,{\tec}のSIOから受信したデータをFT232RLに送信する.
FT232RLはPCとUSBシリアル接続が確立していればデータをPCに送るが,
確立していない場合はデータを無視する.
このようにして,{\tec}のシリアル通信をUSBを経由してPCに中継する.

Bluetoothモジュール(RN4020)は
シリアル通信でデータだけでなくコマンドも受け付ける.
Bluetooth接続が確立さていない状態で{\tac}がRN4020に何か送信すると,
コマンドとして解釈され不具合が生じる可能性がある.
そこで,Bluetooth接続が確立されている場合だけ{\tec}の通信を中継する.
このようにして{\tec}が知らない間に,{\tec}のシリアル通信先が切り換わる.

USBシリアルまたはBluetoothを通してPCから受信したデータに
``\|\033TWRITE\r\n|''の文字列を見つけると,
TWRITEプログラムの通信だと判断する.
TWRITEプログラムが送ってきた{\tec}の機械語プログラムを受信し,
{\tec}のコンソールを操作して{\tec}のメモリに書き込む.

なお,SETAボタンが押された状態で{\tec7}がリセットされた場合は,
OS(``\|kernel.bin|'')を読み込み制御をOSに移す.
この場合は,コンソールから{\tec}が操作できるが,
裏で{\tac}がOSを起動した状態になる.
{\tac}のOS上で{\tec}のプログラムを開発する場合等に使用することを想定している.

\subsection{{\tac}モード}
マイクロSDスロットを確認し,カードが挿入されていればOSを読み込んで起動する.
OSは,マイクロSDカードのFAT16ファイルシステムの
``\|\kernel.bin|''\footnote{
\texttt{.bin}ファイル形式については,「\texttt{Util--}解説書」
(\url{https://github.com/tctsigemura/Util--/raw/master/doc/umm.pdf})の
付録B「ファイルフォーマット」を参照のこと.
}ファイルに格納されている.
IPLはOSに制御を移す前に,
自身が格納されたROM(\texttt{E000h - FFFFh})を切り離しRAMに切り換える.

なお,リセット時にSETAボタンが押されていた場合は,
``\|\kernel.bin|''ファイルの代わりに
``\|\kernel0.bin|''ファイルからOSを読み込む.
カーネルのデバッグ中でも簡単にOSを起動できる.

\subsection{DEMOモード}
「DEMO1モード」,「DEMO2モード」では,
IPLがRN4020とFT232RLの通信を中継する.
USBシリアルで接続したPCから,RN4020の初期設定を行うことができる.
工場出荷時にRN4020のシリアル通信は115,200ボーに設定されているが,
FT232RLのデフォルトは9,600ボーである.
{\tac}がボーレート変換器の役割を果たす.
なお,FT232RL及びRN4020のボーレートは変更してはならない.

\subsection{RESET}
RN4020を工場出荷時の状態に戻す.
通常はシリアル通信でコマンドを送ることでRN4020を初期化できる.
しかし,
間違ってボーレートを変更したり,
ハードウェアフロー制御を有効にしたりすると,
コマンドを送ることができなくなることがある.
そのような場合に,この機能を使用する.

RN4020は,
電源投入後5秒以内に\texttt{WAKE\_HW}ピンを3回以上フリップすることで,
工場出荷時の状態に戻る.
ジャンパーをRESETの設定にして{\tecS}に電源を投入すると,
{\tac}がこの操作を行う.

%=========================================
\section{周辺装置}
\label{io}

{\tac}は,\figref{TaCBlk}に示したように,
コンソール,MMU,割り込みコントローラ,タイマー,
入出力装置などの周辺装置を持っている.
これらには,\figref{tacMap}のI/Oマップに掲載されたポートを通して,
IN,OUT機械語命令でアクセスする.
以下では,周辺装置の使用方法を解説する.
なお,特別な説明がないレジスタ等はリセット時に`0'で初期化される.

%-----------------------------------------
\subsection{タイマー}
\label{timer}
Timer0,Timer1の2チャンネルのインターバルタイマーが使用できる.
タイマーは16ビットのカウンタと16ビットの周期レジスタ等から構成される.

\begin{center}
  \small\begin{tabular}{| r | c | c || c | c |}\hline
    \multirow{2}{*}{番地}
    & \multicolumn{2}{|c||}{IN}
    & \multicolumn{2}{c|}{OUT}
    \\\cline{2-5}
         & 上位バイト & 下位バイト & 上位バイト & 下位バイト
    \\\hline\hline
    00h  &  \multicolumn{2}{| c||}{Timer0カウンタ}
         &  \multicolumn{2}{| c |}{Timer0周期レジスタ} \\\hline
    02h  &  \multicolumn{2}{| c||}{Timer0フラグ}
         &  \multicolumn{2}{| c |}{Timer0制御}     \\\hline
    04h  &  \multicolumn{2}{| c||}{Timer1カウンタ}
         &  \multicolumn{2}{| c |}{Timer1周期レジスタ} \\\hline
    06h  &  \multicolumn{2}{| c||}{Timer1フラグ}
         &  \multicolumn{2}{| c |}{Timer1制御}     \\\hline
  \end{tabular}
\end{center}

\begin{description}
\item[カウンタ]
  カウンタの現在値を読み出すことができる.
  タイマー動作中は1ms毎にカウントアップされ,
  カウンタの値と周期レジスタの値が一致するとゼロにリセットされる.
  リセットされる時,CPUに割り込みを発生する.
  コンソールからCPUを停止している間はカウンタも停止する.
\item[周期レジスタ]
  周期レジスタに書き込んだ値によって,
  カウンタがリセットされる周期が決まる.
  単位はミリ秒である.
\item[フラグ](\texttt{F0000000 00000000})
  カウンタの値と周期レジスタの値が一致すると\texttt{F}に`1'がセットされる.
  同じチャネルのカウンタまたはフラグが読み出されるとリセットされる.
\item[制御](\texttt{I0000000 0000000S})
  \texttt{I}が割り込み許可ビット,
  \texttt{S}がカウンタのスタート/ストップ(`1'/`0')を制御する.
  制御ワードに書き込みを行うとカウンタがリセットされるので,
  カウントは必ずリセット状態から開始される.
\end{description}

%-----------------------------------------
\subsection{FT232RL(シリアルI/O)}
USBシリアル変換IC(FT232RL)を通してPCと通信を行うことができる.
変調速度は9,600ボーに固定されており変更することはできない.
送信・受信の両方で割り込みを発生することができる.

\begin{center}
  \small\begin{tabular}{| r | c | c || c | c |}\hline
    \multirow{2}{*}{番地}
    & \multicolumn{2}{|c||}{IN}
    & \multicolumn{2}{c|}{OUT}
    \\\cline{2-5}
         & 上位バイト & 下位バイト & 上位バイト & 下位バイト
    \\\hline\hline
    08h  &  00 & 受信データ
         &  -  & 送信データ \\\hline
    0Ah  &  00 & ステータス
         &  -  & 制御 \\\hline
  \end{tabular}
\end{center}

\begin{description}
\item[受信データ]
  FT232RLから受信した1バイトのデータを読み出す.
\item[送信データ]
  FT232RLへ送信する1バイトのデータを書き込む.
\item[ステータス](\texttt{TR00 0000})
  送信回路に送信データを書き込み可能なとき\texttt{T}が`1'になる.
  受信回路に受信済みデータがあり読み出し可能なとき\texttt{R}が`1'になる.
\item[制御](\texttt{TR00 0000})
  \texttt{T}を`1'にすると次の送信データが
  書き込み可能になる度に割込みが発生する.
  \texttt{R}を`1'にすると次の受信データが
  読み込み可能になる度に割込みが発生する.
\end{description}

%-----------------------------------------
\subsection{TeC(シリアルI/O)}
TeCとシリアルデータ通信ができる.
変調速度は9,600ボーに固定されており変更することはできない.
送信・受信の両方で割り込みを発生することができる.

\begin{center}
  \small\begin{tabular}{| r | c | c || c | c |}\hline
    \multirow{2}{*}{番地}
    & \multicolumn{2}{|c||}{IN}
    & \multicolumn{2}{c|}{OUT}
    \\\cline{2-5}
         & 上位バイト & 下位バイト & 上位バイト & 下位バイト
    \\\hline\hline
    0Ch  &  00 & 受信データ
         &  -  & 送信データ \\\hline
    0Eh  &  00 & ステータス
         &  -  & 制御 \\\hline
  \end{tabular}
\end{center}

\begin{description}
\item[受信データ]
  TeCから受信した1バイトのデータを読み出す.
\item[送信データ]
  TeCへ送信する1バイトのデータを書き込む.
\item[ステータス](\texttt{TR00 0000})
  送信回路に送信データを書き込み可能なとき\texttt{T}が`1'になる.
  受信回路に受信済みデータがあり読み出し可能なとき\texttt{R}が`1'になる.
\item[制御](\texttt{TR00 0000})
  \texttt{T}を`1'にすると次の送信データが
  書き込み可能になる度に割込みが発生する.
  \texttt{R}を`1'にすると次の受信データが
  読み込み可能になる度に割込みが発生する.
\end{description}

%-----------------------------------------
\subsection{マイクロSDホストコントローラ}
マイクロSDとメモリの間でセクタ単位の読み書きができる.

\begin{center}
  \small\begin{tabular}{| r | c | c || c | c |}\hline
    \multirow{2}{*}{番地}
    & \multicolumn{2}{|c||}{IN}
    & \multicolumn{2}{c|}{OUT}
    \\\cline{2-5}
         & 上位バイト & 下位バイト & 上位バイト & 下位バイト
    \\\hline\hline
    10h  &  00 & ステータス
         &  -  & 制御 \\\hline
    12h  &  \multicolumn{2}{|c||}{メモリアドレス}
         &  \multicolumn{2}{|c| }{メモリアドレス}     \\\hline
    14h  &  \multicolumn{2}{|c||}{セクタアドレス上位}
         &  \multicolumn{2}{|c| }{セクタアドレス上位} \\\hline
    16h  &  \multicolumn{2}{|c||}{セクタアドレス下位}
         &  \multicolumn{2}{|c| }{セクタアドレス下位} \\\hline
  \end{tabular}
\end{center}

\begin{description}
\item[ステータス](\texttt{IE00 0000})
  ホストコントローラの状態を表す.
  \texttt{I}はアイドル状態を表す.
  \texttt{E}はエラーが発生したことを表す.
\item[制御](\texttt{E000 0IRW})
  \texttt{I}に`1'を書き込むと,
  マイクロSDをSPIモードに切り換え使用できるように初期化する動作を開始する.
  \texttt{R}に`1'を書き込むとマイクロSDから1セクタ読み込む動作を開始する.
  \texttt{W}に`1'を書き込むとマイクロSDに1セクタ書き込む動作を開始する.
  \texttt{E}を`1'にすると上記の動作が完了したとき
  割り込みが発生するようになる.
\item[メモリアドレス]
  セクタから読み込んだデータ,または,セクタに書き込むデータを
  格納するバッファのメモリアドレスを設定する.
  ホストコントローラはCPUの力を借りることなく,
  メモリとマイクロSDの間でデータの転送を行う.
  バッファサイズは512,アドレスは偶数でなければならない.
\item[セクタアドレス上位]
  データを読み書きするセクタのLBA(Logical Block Addressing)方式の
  32ビットのアドレスの上位16ビットである.
\item[セクタアドレス下位]
  LBA方式の32ビットのアドレスの下位16ビットである.
\end{description}

%-----------------------------------------
\subsection{入出力ポート他}
{\tecS}の入出力ポート\footnote{\figref{TeC7Photo}参照のこと.}に
パラレルデータを入出力する.

\begin{center}
  \small\begin{tabular}{| r | c | c || c | c |}\hline
    \multirow{2}{*}{番地}
    & \multicolumn{2}{|c||}{IN}
    & \multicolumn{2}{c|}{OUT}
    \\\cline{2-5}
         & 上位バイト & 下位バイト & 上位バイト & 下位バイト
    \\\hline\hline
    18h  &  00 & 入力ポート
         &  -  & 出力ポート \\\hline
    1Ah  &  00 & 00
         &  -  & ADC参照電圧 \\\hline
    1Ch  &  00 & 00
         &  -  & 出力ポート上位 \\\hline
    1Eh  &  00 & モード
         &  - & - \\\hline
  \end{tabular}
\end{center}

\begin{description}
\item[入力ポート]
  入出力ポートの\texttt{I7}〜\texttt{I0}\footnote{
    \figref{TeC7Photo}の入出力ポートコネクタ左にピン配置が印刷されている.
  }の8ビットの入力値を読み取る
\item[出力ポート]
  入出力ポートの\texttt{O7}〜\texttt{O0}の8ビットに出力する値を設定する.
\item[ADC参照電圧]
  プリント基板上のADコンバータ回路の参照電圧を決定する.
  {\tac}モードでは,ADコンバータはソフトウェアで制御する必要がある.
  リセット時は\texttt{0x80}がセットされる.
\item[出力ポート上位](\texttt{M000 VVVV})
  \texttt{M}を`1'にすると
  入力ポートの\texttt{I7}〜\texttt{I4}が出力ポートに切り換わる.
  \texttt{M}と同時に書き込んだ\texttt{VVVV}の4ビットが,
  \texttt{I7}〜\texttt{I4}に出力される.
\item[モード](\texttt{0000 0MMM})
  {\tecS}の動作モード\footnote{詳しくは\ref{tec7mode}を参照のこと.}を
  \texttt{MMM}の3ビットから知ることができる.
  \texttt{MMM}の意味は,TeCモード(\texttt{000}),TaCモード(\texttt{001}),
  DEMO1モード(\texttt{010}),DEMO2モード(\texttt{011}),
  RN4020リセット(\texttt{111})である.
\end{description}

%-----------------------------------------
\subsection{SPIインタフェース}

\figref{Spi}にSPIインタフェースの概略図を示す.
入出力ポートの\texttt{O1}ビットにSCLK,\texttt{O0}ビットにSOを出力し,
\texttt{I6}ビットをSIとして入力するSPIインタフェースである.
出力の2ビットは出力ポートの下位2ビットとXORをとっているので,
出力ポートの値で極性を変更することができる.
SPIで接続した周辺LSIがSCLKを誤って認識しないように,
出力ポートの値を変更するときは,
CSをインアクティブにしなければならない.
シフトレジスタにデータが書かれると動作を開始する.
(データを受信する際も,シフトレジスタにデータを書き込む.)

\myfigure{tbp}{scale=.8}{Fig/Spi.pdf}{SPIインタフェースの概略}{Spi}

\begin{center}
  \small\begin{tabular}{| r | c | c || c | c |}\hline
    \multirow{2}{*}{番地}
    & \multicolumn{2}{|c||}{IN}
    & \multicolumn{2}{c|}{OUT}
    \\\cline{2-5}
         & 上位バイト & 下位バイト & 上位バイト & 下位バイト
    \\\hline\hline
    20h  &  00 & シフトレジスタ
         &  -  & シフトレジスタ \\\hline
    22h  &  00 & ステータス
         &  -  & SCLK周期       \\\hline
  \end{tabular}
\end{center}

\begin{description}
\item[シフトレジスタ]
  8ビットのデータを読み書きする.
  データが書き込まれるとデータが1ビットずつSOに出力される.
  同時にSIからデータが1ビットずつシフトレジスタに読み込まれる.
\item[ステータス](\texttt{0000 000B})
  シフトレジスタが動作中に\texttt{B(Busy)}ビットが`1'になる.
\item[SCLK周期]
  SPIのSCLKの周波数を決める.
  SCLK周波数は96kHz〜24.576MHzの範囲で細かく設定できる.
  書き込む値を$N$とすると周波数は次の式で計算できる.

  \centerline{$SCLK周波数 = 24.576 \div ( N + 1 ) MHz$}
\end{description}

\begin{center}
  \small\begin{tabular}{ r | r }\hline\hline
  \multicolumn{1}{c|}{N} & \multicolumn{1}{|c}{SCLK周波数(MHz)} \\\hline
  0   & 24.576 \\
%  1   & 12.288 \\
  3   &  6.144 \\
%  7   &  3.072 \\
%  15  &  1.536 \\
  31  &  0.768 \\
%  63  &  0.384 \\
%  127 &  0.192 \\
  255 &  0.096 \\
  \end{tabular}
\end{center}

%-----------------------------------------
\subsection{入力ポート割り込み}
入出力ポートの\texttt{I7}〜\texttt{I0}を監視し,
入力が変化した時に割り込みを発生することができる.
監視対象ビット全ての論理和をとり,
結果が`0'から`1'に変化する時に割り込みが発生する.

\begin{center}
  \small\begin{tabular}{| r | c | c || c | c |}\hline
    \multirow{2}{*}{番地}
    & \multicolumn{2}{|c||}{IN}
    & \multicolumn{2}{c|}{OUT}
    \\\cline{2-5}
         & 上位バイト & 下位バイト & 上位バイト & 下位バイト
    \\\hline\hline
    24h  &  00 & 00
         &  -  & MASK \\\hline
    26h  &  00 & 00
         &  -  & XOR \\\hline
  \end{tabular}
\end{center}

\begin{description}
\item[MASK]
  入力ポートの監視するビットを設定する.
  `1'を設定したビットが監視対象になる.
\item[XOR]
  ここに設定した値は監視するビットと排他的論理和をとるために使用する.
\end{description}

複数のビットを同時に監視する際は,
「監視対象ビット全ての論理和をとり,
結果が`0'から`1'に変化する時に割り込みが発生する.」ことを考慮し,
適切な順序で\texttt{MASK}と\texttt{XOR}を操作する必要がある.

%-----------------------------------------
\subsection{RN4020アダプタ}
Bluetoothモジュール(RN4020)を接続するインタフェースである.
TeC7aにはない.

\begin{center}
  \small\begin{tabular}{| r | c | c || c | c |}\hline
    \multirow{2}{*}{番地}
    & \multicolumn{2}{|c||}{IN}
    & \multicolumn{2}{c|}{OUT}
    \\\cline{2-5}
         & 上位バイト & 下位バイト & 上位バイト & 下位バイト
    \\\hline\hline
    28h  &  00 & 受信データ
         &  -  & 送信データ \\\hline
    2Ah  &  00 & ステータス
         &  -  & 制御 \\\hline
    2Ch  &  00 & 00
         &  -  & コマンド \\\hline
    2Eh  &  00 & 接続状況
         &  -  & 接続状況 \\\hline
  \end{tabular}
\end{center}

\begin{description}
\item[受信データ]
  RN4020から受信した1バイトのデータを読み出す.
\item[送信データ]
  RN4020へ送信する1バイトのデータを書き込む.
\item[ステータス](\texttt{TR00 0000})
  送信回路に送信データを書き込み可能なとき\texttt{T}が`1'になる.
  受信回路に受信済みデータがあり読み出し可能なとき\texttt{R}が`1'になる.
\item[制御](\texttt{TR00 0000})
  \texttt{T}を`1'にすると次の送信データが
  書き込み可能になる度に割込みが発生する.
  \texttt{R}を`1'にすると次の送信データが
  読み込み可能になる度に割込みが発生する.
\item[コマンド](\texttt{0000 FHCS})
  \texttt{F}を`1'にするとRN4020と{\tac}間のシリアル通信の
  ハードウェアフロー制御が有効になる.
  \texttt{H}はRN4020の\|WAKE_HW|ピンを制御する.
  \texttt{C}はRN4020の\|CMD/MLDP|ピンを制御する.
  \texttt{S}はRN4020の\|WAKE_SW|ピンを制御する.
  \texttt{S}にはリセット時に`1'が設定される.
\item[接続状況](\texttt{RRRR RRRC})
  \texttt{R}はリセットされないメモリである.
  RESETボタンが押され{\tac}が再起動しても以前の状態を維持する.
  \texttt{C}の意味は,TeC7b,TeC7cとTeC7dで異なる.
  \begin{itemize}
  \item TeC7b,TeC7cの場合,\texttt{C}はRESETされない1ビットのメモリである.
    IPLプログラムとOSはRN4020からの受信データを監視し,
    BlueTerminal(\url{https://github.com/tctsigemura/BlueTerminal})との
    接続確立・切断が発生したことを判定し\texttt{C}に接続状態を書き込む.
  \item TeC7dの場合,\texttt{C}はRN4020の
    \texttt{CONNECTION LED}ピンの状態を反映する.
    このビットへの書き込みはできない.(無視される.)
  \end{itemize}
\end{description}

%-----------------------------------------
\subsection{TeCアダプタ}
{\tec}モードで動作中に,
{\tac}のプログラムで{\tec}のコンソールを操作できる.
{\tac}のIPLがTWRITEの通信内容に応じて{\tec}を操作するために使用している
\footnote{IPLとTWRITEについては\ref{ipl}を参照すること.}.
以下でポートに書き込むビット値は,
`1'がスイッチを上に倒した状態,または,ボタンを押した状態を表す.

\begin{center}
  \small\begin{tabular}{| r | c | c || c | c |}\hline
    \multirow{2}{*}{番地}
    & \multicolumn{2}{|c||}{IN}
    & \multicolumn{2}{c|}{OUT}
    \\\cline{2-5}
         & 上位バイト & 下位バイト & 上位バイト & 下位バイト
    \\\hline\hline
    30h  &  00 & データランプ
         &  -  & -              \\\hline
    32h  &  00 & 00
         &  -  & データスイッチ \\\hline
    34h  &  00 & 00
         &  -  & 機能スイッチ   \\\hline
    36h  &  00 & スイッチ状態
         &  -  & 制御           \\\hline
  \end{tabular}
\end{center}

\begin{description}
\item[データランプ]
  {\tec}コンソールのデータランプの表示を読み取ることができる.
  {\tec}のメモリを読み出す{\tac}プログラムを作成可能にする.
\item[データスイッチ]
  コンソールのデータスイッチの代わりに{\tec}に入力する値を設定する.
\item[機能スイッチ](\texttt{ABCD EFGH})
  {\tec}コンソールの一番下の八つのスイッチを操作する.
  各ビットの意味は下の表の通りである.
\item[スイッチ状態](\texttt{0000 00RS})
  \texttt{R}はRESETボタン,
  \texttt{S}はSETAボタンが押されていることを表す.
\item[制御](\texttt{I000 0JKL})
  \texttt{I}は\figref{TeC7Blk}のMUX1を操作し,
  TeCアダプタの機能を有効にするビットである.
  \texttt{J},\texttt{K},\texttt{L}には,
  RESET等のボタンを操作するための値を設定する.
  各ビットの意味は下の表の通りである.
\end{description}

\begin{center}
  \small\begin{tabular}{c | l}\hline\hline
  ビット     & \multicolumn{1}{|c}{スイッチ}\\\hline
  \texttt{A} & BREAK \\
  \texttt{B} & STEP  \\
  \texttt{C} & RUN   \\
  \texttt{D} & STOP  \\
  \texttt{E} & SETA  \\
  \texttt{F} & INCA  \\
  \texttt{G} & DECA  \\
  \texttt{H} & WRITE \\
  \texttt{I} & 制御を有効化 \\
  \texttt{J} & RESET \\
  \texttt{K} & ←  \\
  \texttt{L} & →  \\
  \end{tabular}
\end{center}

%-----------------------------------------
\subsection{コンソール}
{\tac}モードでプログラム実行中は,
コンソールをプログラムの入出力装置として使用できる.

\begin{center}
  \small\begin{tabular}{| r | c | c || c | c |}\hline
    \multirow{2}{*}{番地}
    & \multicolumn{2}{|c||}{IN}
    & \multicolumn{2}{c|}{OUT}
    \\\cline{2-5}
         & 上位バイト & 下位バイト & 上位バイト & 下位バイト
    \\\hline\hline
    F8h  &  00 & データSW
         &  \multicolumn{2}{| c |}{データレジスタ} \\\hline
    FAh  &  \multicolumn{2}{| c||}{アドレスレジスタ}
         &  - & - \\\hline
    FCh  &  00 & ロータリーSW
         &  - & - \\\hline
    FEh  &  00 & 機能レジスタ
         &  - & - \\\hline
  \end{tabular}
\end{center}

\begin{description}
\item[データSW]
  データSW(8個のトグルスイッチ)の現在の状態を読むことができる.
\item[データレジスタ]
  アドレス・データランプ(合計16個のLED)のON/OFFを制御できる.
\item[アドレスレジスタ]
  \ref{rotarySW}で説明したMA(Memory Address register)の値を読むことができる.
\item[ロータリーSW]
  ロータリースイッチの位置(G0=0,G1=1 ... MA=17)を読むことができる.
\item[機能レジスタ]
  このポートからWRITEスイッチが押されたことを知ることができる.
\end{description}


