\chapter{{\tac}のアーキテクチャ}

{\tec}のアーキテクチャは「TeC教科書」\footnote{
\url{https://github.com/tctsigemura/TecTextBook/raw/master/tec.pdf}}
で詳しく説明されているので,
ここでは{\tac}のアーキテクチャについて簡単に説明する.

%-----------------------------------------
\section{CPUの概要}
{\tac}で使用できるデータの形式,
アドレス空間,
実行モード,
CPU内部のレジスタ構成,
機械語命令,割込みと例外について説明する.

\subsection{データ形式}
\figref{tacData}の「データ形式」に{\tac}が扱うことができるデータを示す.
16ビットの整数データと,16ビットのアドレスデータの他に,
8ビットの整数データを扱うことができる.
16ビットのデータはCPUの内部でもメモリやI/Oでも使用できる.
8ビットデータはメモリ上の配列に格納することだけを想定している.

\subsection{アドレス空間}
\figref{tacData}の「メモリ空間」,「I/O空間」に
{\tac}のメモリとI/Oのアドレス空間を示す.
メモリ空間もI/O空間もバイト単位でアドレスが割り付けてある.
しかし,I/O空間をバイト単位でアクセスする機械語命令を現在のTaCは備えていない
\footnote{以前のTaCはI/O空間もバイト単位でアクセスできたので,
バイト単位でアドレスが割り付けられている.}.
メモリやI/Oの16ビットデータにアクセスする場合は偶数番地を用いる.
8ビットデータはメモリの読み書きだけに使用できる.
メモリの8ビットデータにアクセスする場合は,
CPUレジスタの下位8ビットだけが使用される.

\subsection{実行モード}
{\tac}は「特権モード」,「ユーザモード」,「I/O特権モード」の
三つの実行モードを持っている.

\begin{description}
\item[特権モード]
全ての機械語命令が実行できるモードである.
OSカーネルは特権モードで実行される.
\item[ユーザモード]
実行モードを変更したり,
ハードウェアの状態を変更したりする\emph{特権命令}を実行することができない.
通常,ユーザプログラムはユーザモードで実行される.
\item[I/O特権モード]
IN,OUT機械語命令が実行できるユーザモードである.
入出力ポートに接続したオプションのハードウェア\footnote{
このようなハードウェアはOSによってサポート・管理されない.
}を使用するプリケーションを実行するために用意されている.
\end{description}

\subsection{CPUレジスタとPSW}
\figref{tacData}の「レジスタ構成」にCPU内部のレジスタなどを示す.
レジスタはどれも16ビット幅である.

\subsubsection{CPUレジスタ}
CPUレジスタは,
汎用のG0(General register 0)からG11,
フレームポインタとして使用するFP(Frame Pointer),
特権モード用のスタックポインタSSP(System Stack Pointer),
ユーザモード(I/O特権モードも含む)用の
スタックポインタUSP(User Stack Pointer)からなる.
これらは全て計算用にもアドレス用にも使用できる.
FP,SSP,USPは,以下に説明する特別な意味も持っている.

\subsubsection{フレームポインタ(Frame Pointer)}
フレームポインタ(FP)はCPUレジスタの一つである.
フレームポインタ相対アドレッシングモードで使用できる.
このアドレッシングモードを用いると,
スタックフレーム内のローカル変数や関数引数へ,
1ワード(2バイト)の機械語命令でアクセスできる.

\subsubsection{スタックポインタ(Stack Pointer)}
スタックポインタ(SP)もCPUレジスタの一つである.
{\tac}は特権モード用(SSP),
ユーザモード(I/O特権モード含む)用(USP)の
二本のスタックポインタを持っている.
SSPは特権モードでSPの位置にマップされ,
OSカーネル用のスタックポインタとして使用される.
USPはユーザモード(I/O特権モード含む)でSPの位置にマップされ,
ユーザプログラムのスタックポインタとして使用される.
USPはG14として常時マップされており,
特権モードでもUSPをアクセスすることができる.

\subsubsection{PSW(Program Status Word)}
PSWはPC(Program Counter)とFLAGからなる.
FLAGには,計算結果で変化する
\texttt{V(oVerflow)},
\texttt{C(Carry)},
\texttt{S(Sign)},
\texttt{Z(Zero)}と,
割込み許可\texttt{E(Enable interrupt)},
特権モード\texttt{P(Privilege)},
I/O特権モード\texttt{I(I/O Privilege)},
ユーザ定義\texttt{U(User defined)}の各ビットがある.

FLAGはG15として普通の機械語命令で扱うことも可能であるが,
ユーザモードでは\texttt{E},\texttt{P},\texttt{I}の各ビットは変化しない.

\subsection{機械語命令}
\figref{tacInst}に{\tac}の機械語命令の一覧表を示す.
HALTは特権モードでしか使用できない\emph{特権命令}である.
IN,OUTは特権モードとI/O特権モードで使用できる命令である.
これらの命令を非特権モードで実行すると「特権違反例外」が発生する.
SVC命令はシステムコールを発行するために「SVC例外」を発生する.

ほとんどの転送命令と計算命令で8種類のアドレッシング・モードが使用できる.
Direct,Indexed,Immediateの
三つのアドレッシング・モードを使用する場合は2ワードの機械語命令になる.
他のアドレッシング・モードの場合は1ワード命令である.

Byte Register Indirect アドレッシング・モードだけが,
メモリの8ビットデータをアクセスする.
Byte Register Indirect アドレッシング・モードの
ST命令は,CPUレジスタの下位8ビットをメモリに書き込む.
これら以外の命令は,メモリから読み出した8ビットデータの上位に
\|00h|を付加した16ビットデータを使用する.

\subsection{割込み(Interrupt)と例外(Exception)}
{\tac}はベクタ方式(ベクタは\|FFE0h|番地〜)の割込み・例外機構を備えている.
割込み・例外の種類は\tabref{inter}に示す16種類である.
ゼロ除算や特権違反のような
ソフトウェアに起因する割込みを「例外(Exception)」と呼ぶ.
「TLBミス」から「SVC」までの6種類が「例外」 それ以外が「割込み」である.
「割込み」の許可と禁止はFLAGの\texttt{E}ビットを操作することで行う.
「例外」の発生は禁止できない.

割込み(例外)が発生すると次の順で割込み(例外)処理が行われる.
\begin{enumerate}
\item CPU内部の一時レジスタに\emph{FLAGのコピー}が作られる.
\item FLAGが変更され,
割込みが禁止(\texttt{E=0})の特権モード(\texttt{P=1})になる.
\item PCと\emph{FLAGのコピー}が順にカーネルスタックにPUSHされる.
\item PCに割込み(例外)ハンドラの開始番地がロードされ,
ハンドラの実行が開始される.
\end{enumerate}

\begin{mytable}{btp}{割込み・例外の種類と意味}{inter}
  \begin{tabular}{ r  l | l }\hline\hline
    \multicolumn{2}{c}{割込み・例外} &
    \multicolumn{1}{|c}{意 味} \\\hline
    0 & Timer0      & ハードウェアタイマー0に設定された時刻になった.\\
    1 & Timer1      & ハードウェアタイマー1に設定された時刻になった.\\
    2 & RN4020受信  & Bluetoothモジュールから1バイトのデータを受信した.\\
    3 & RN4020送信  & Bluetoothモジュールへ1バイトのデータを送信し終えた. \\
    4 & FT232RL受信 & USBシリアル変換ICから1バイトのデータを受信した.\\
    5 & FT232RL送信 & USBシリアル変換ICへ1バイトのデータを送信し終えた. \\
    6 & TeC受信     & TeCから1バイトのデータを受信した. \\
    6 & TeC送信     & TeCへ1バイトのデータを送信し終えた. \\
    8 & マイクロSD  & マイクロSDのホストコントローラが
                      コマンドを実行し終えた.\\
    9 & PIO         & 入出力ポートの監視中のビットに変化があった. \\
    10& TLB ミス    & MMU有効時にTLBに必要なエントリが見つからない. \\
    11& メモリ保護違反 & 奇数アドレスでワードデータをアクセスした.または,\\
      &                & ページの保護モード(RWX)に
                         一致しないアクセスがされた. \\
    12& ゼロ除算    & 割り算機械語命令で「÷ 0」が実行された. \\
    13& 特権違反    & 不適切な実行モードで特権命令が実行された. \\
    14& 未定義命令  & {\tac}の機械語として解釈できない命令を実行した. \\
    15& SVC         & SVC 機械語命令が実行された. \\
  \end{tabular}
\end{mytable}

%-----------------------------------------
\section{メモリマップとI/Oマップ}
メモリ空間とI/O空間は8ビット毎にアドレス付けされている.
メモリは8ビットデータ,16ビットデータのどちらも読み書きできる.
I/Oは16ビットデータの読み書きしかできない.
メモリの場合は機械語命令のアドレッシング・モードによって,
8ビットデータと16ビットデータの区別をする.
16ビットデータは偶数アドレスを指定してアクセスしなければならない.

\subsection{メモリ空間}
\figref{tacMap}の「メモリマップ」に{\tac}のメモリマップを示す.
{\tac}のメモリ空間は\|0000h|から\|FFFFh|の64KiBである.
16ビットデータは偶数アドレスからの2バイトに配置され,
偶数アドレスを指定してアクセスする.
8ビットデータにアクセスするには,
Byte Register Indirect モードを用いる.
その他のアドレッシング・モードは,
16ビットデータをアクセスするために用いる.

リセット時に,\|E000h|から\|FFFFh|にIPL(ROM)が配置される.
{\tac}モードでは,IPLはマイクロSDからOSを読み出して起動する.
その他のモードでは,IPLが{\tec}の通信を中継する等の機能を果たす.
IPLはOSを読みだしたらIPL(ROM)を切り離しメモリ空間全体をRAMにした後,
OSに制御を渡す.
IPL(ROM)が切り離された後,
\|FFE0h|から\|FFFFh|は割込みベクタ領域になる.
16種類の割込み・例外ハンドラの入口番地をOSがセットする.

\subsection{I/O空間}
\figref{tacMap}の「I/Oマップ」に{\tac}のI/Oマップを示す.
{\tac}のI/O空間は\|00h|から\|FFh|の128ワードである.
I/O空間のアドレス幅は8ビットだが,
IN,OUT機械語命令ではI/Oアドレスが16ビットで表現される.
I/Oアドレスの上位8ビットは\|00h|になるようにする.
上位8ビットが\|00h|以外になった場合の動作は保証されない.
メモリ空間と異なり16ビットデータの読み書きしかできない.

%-----------------------------------------
\section{MMU(Memory Management Unit)}
ページング方式のMMUが使用できる.
MMUが働くのはユーザモードで実行中だけである.
MMUの制御は
\figref{tacMap}の「I/Oマップ」に示すポートをIN,OUT機械語命令で操作して行う.
以下では,MMUに関係のあるポートについて説明する.

\begin{center}
  \small\begin{tabular}{| r | c | c || c | c |}\hline
    \multirow{2}{*}{番地}
    & \multicolumn{2}{c||}{IN}
    & \multicolumn{2}{c|}{OUT}
    \\\cline{2-5}
         & 上位バイト & 下位バイト & 上位バイト & 下位バイト
    \\\hline\hline
    80h  & 00 & TLB[0]上位8bit & - & TLB[0]上位8bit \\\hline
    82h  & \multicolumn{2}{c||}{TLB[0]下位16bit}
         & \multicolumn{2}{c|}{TLB[0]下位16bit} \\\hline
    84h  & 00 & TLB[1]上位8bit & - & TLB[1]上位8bit \\\hline
    86h  & \multicolumn{2}{c||}{TLB[1]下位16bit}
         & \multicolumn{2}{c|}{TLB[1]下位16bit} \\\hline
    ...  & \multicolumn{2}{c||}{...}
         & \multicolumn{2}{c|}{...}   \\\hline
    9Ch  & 00 & TLB[7]上位8bit & - & TLB[7]上位8bit \\\hline
    9Eh  & \multicolumn{2}{c||}{TLB[7]下位16bit}
         & \multicolumn{2}{c|}{TLB[7]下位16bit} \\\hline
    A0h  &  00 & 00
         &  -  & IPL切離し \\\hline
    A2h  &  \multicolumn{2}{c||}{違反アドレス}
         &  -  & MMU有効化 \\\hline
    A4h  &  00 & 違反原因
         &  -  & - \\\hline
    A6h  &  00 & ページ番号
         &  -  & - \\\hline
  \end{tabular}
\end{center}

\subsection{違反アドレス}
  メモリ保護違反が発生した時,原因となった論理アドレスが記録される.

\subsection{違反原因(\texttt{0000 00BV})}
  メモリ保護違反が発生した時,
  奇数アドレスを用いたワードアクセス(Bad Address)の場合
  \texttt{B}ビットが`1'になる.
  ページの保護モード違反(Memory Violation)の場合
  \texttt{V}ビットが`1'になる.
  これらのビットはCPUがIN命令でA4h番地を読むとクリアされる.

\subsection{ページ番号}
  ページ番号がTLBでヒットしなかった場合「TLB ミス例外」が発生する.
  その際,例外の原因となったページ番号が記録される.

\subsection{IPL切離し(\texttt{0000 000I})}
  \texttt{I}に`1'を書き込むと,
  物理メモリ空間最後の8KiBに配置されたIPL(ROM)が切り離されRAMに置き換わる.
  これによりメモリ空間64KiB全てがRAMになる.
  通常,切り離しの操作はIPL自身がOSをロードした後で自動的に行う.
  IPL(ROM)がマップされている時でもコンソールからはRAMが見えるので,
  IPLを最後まで実行していない場合は注意が必要である.

\subsection{MMU有効化(\texttt{0000 000E})}
  \texttt{E}に`1'を書き込むとMMUが有効になる.
  MMUが有効になるとCPUの実行モードが「I/O特権モード」,「ユーザモード」の時,
  ページからフレームへの変換($p \to f$変換)が行われる.

\subsection{TLB(\texttt{TLB[0]〜TLB[7])}}
  TLBはページ番号で検索されフレーム番号を出力し$p \to f$変換を行う.
  8エントリのTLBをIN,OUT機械語命令で参照・操作することができる.
  8つのTLBエントリはI/O空間の80h番地から9Eh番地までの範囲に配置される.
  1つのTLBエントリは24bitであるので,
  上位8bitと下位16bitに分けてアクセスする.
  以下ではTLBエントリの各ビット(フィールド)の意味を説明する.

  \begin{center}
  \small\begin{tabular}{| c || c |c|c|c|c|c|c| c |}\hline
    \multicolumn{9}{|c|}{TLBエントリの構成} \\\hline
    I/Oポート   &上位8bit&\multicolumn{7}{c|}{下位16bit} \\\hline
    ビット番号  &23 - 16&15&14&13&12&11&10-8&7 - 0      \\\hline
    ビットの意味& ページ番号&
    \|V|&\|U|&\|U|&\|R|&\|D|&\|RWX|&フレーム番号\\\hline
  \end{tabular}
  \end{center}

\subsubsection{ページ番号(\texttt{p})} 
エントリが管理するページの番号を設定する.
TLBエントリをページ番号で検索し,
該当ページを管理するエントリが見つからない場合,「TLB ミス例外」が発生する.

\subsubsection{\texttt{V}(Valid)}
エントリが有効かどうかを表す.
TLBを検索する際,$V=0$のエントリーは無視される.

\subsubsection{\texttt{U}(Undefined)}
ハードウェアでは使用してないビットである.
1ビットの記憶装置としてOSが自由に使用できる.

\subsubsection{\texttt{R}(Reference)}
OSがクリアしたあとユーザプログラムがページを参照すると`1'になる.
OSがページの参照に関する統計情報を取得するために使用する.

\subsubsection{\texttt{D}(Dirty)}
OSがクリアしたあとユーザプログラムがページに書き込みを行うと`1'になる.
OSがページをスワップアウトする必要があるか判断するために使用する.

\subsubsection{\texttt{RWX}(Read/Write/eXecute)}
ページの保護モードを3ビットの組み合わせで表現する.
機械語セグメントには\texttt{101},
データセグメントやスタックセグメントには\texttt{110},
読み出し専用のデータセグメントには\texttt{100}をセットする.
CPUが保護モードで許可されない種類のアクセスを行った場合,
「メモリ保護違反例外」が発生する.

\subsubsection{フレーム番号(\texttt{f})}
ページ番号に対応するフレーム番号を設定する.
MMUはページ番号がヒットしたエントリのフレーム番号を変換結果として出力する.

%-----------------------------------------
\section{IPLプログラム}
\label{ipl}
{\tac}はリセットされると自動的にIPLプログラム\footnote{
IPLのソースコードは
\url{https://github.com/tctsigemura/TeC7/tree/master/TaC/Ipl}
に公開されている.
}の実行を開始する.
IPLの第一の役割は,マイクロSDからOSを読出し起動することである.
しかし,{\tecS}の動作モード(\ref{tec7mode}参照)によっては,
{\tec}の補助(\ref{tec7assist}参照)を行う.
以下では動作モード毎にIPLの役割を説明する.

\subsection{{\tec}モード}
USBシリアル変換IC(FT232RL)から受信したデータを{\tec}のSIOへ送信する.
また,{\tec}のSIOから受信したデータをFT232RLに送信する.
FT232RLはPCとUSBシリアル接続が確立していればデータをPCに送るが,
確立していない場合はデータを無視する.
このようにして,{\tec}のシリアル通信をUSBを経由してPCに中継する.

Bluetoothモジュール(RN4020)は
シリアル通信でデータだけでなくコマンドも受け付ける.
Bluetooth接続が確立さていない状態で{\tac}がRN4020に何か送信すると,
コマンドとして解釈され不具合が生じる可能性がある.
そこで,Bluetooth接続が確立されている場合だけ{\tec}の通信を中継する.
このようにして{\tec}が知らない間に,{\tec}のシリアル通信先が切り換わる.

USBシリアルまたはBluetoothを通してPCから受信したデータに
``\|\033TWRITE\r\n|''の文字列を見つけると,
TWRITEプログラムの通信だと判断する.
TWRITEプログラムが送ってきた{\tec}の機械語プログラムを受信し,
{\tec}のコンソールを操作して{\tec}のメモリに書き込む.

なお,SETAボタンが押された状態で{\tec7}がリセットされた場合は,
OS(``\|kernel.bin|'')を読み込み制御をOSに移す.
この場合は,コンソールから{\tec}が操作できるが,
裏で{\tac}がOSを起動した状態になる.
{\tac}のOS上で{\tec}のプログラムを開発する場合等に使用することを想定している.

\subsection{{\tac}モード}
マイクロSDスロットを確認し,カードが挿入されていればOSを読み込んで起動する.
OSは,マイクロSDカードのFAT16ファイルシステムの
``\|\kernel.bin|''\footnote{
\texttt{.bin}ファイル形式については,「\texttt{Util--}解説書」
(\url{https://github.com/tctsigemura/Util--/raw/master/doc/umm.pdf})の
付録B「ファイルフォーマット」を参照のこと.
}ファイルに格納されている.
IPLはOSに制御を移す前に,
自身が格納されたROM(\texttt{E000h - FFFFh})を切り離しRAMに切り換える.

なお,リセット時にSETAボタンが押されていた場合は,
``\|\kernel.bin|''ファイルの代わりに
``\|\kernel0.bin|''ファイルからOSを読み込む.
カーネルのデバッグ中でも簡単にOSを起動できる.

\subsection{DEMOモード}
「DEMO1モード」,「DEMO2モード」では,
IPLがRN4020とFT232RLの通信を中継する.
USBシリアルで接続したPCから,RN4020の初期設定を行うことができる.
工場出荷時にRN4020のシリアル通信は115,200ボーに設定されているが,
FT232RLのデフォルトは9,600ボーである.
{\tac}がボーレート変換器の役割を果たす.
なお,FT232RL及びRN4020のボーレートは変更してはならない.

\subsection{RESET}
RN4020を工場出荷時の状態に戻す.
通常はシリアル通信でコマンドを送ることでRN4020を初期化できる.
しかし,
間違ってボーレートを変更したり,
ハードウェアフロー制御を有効にしたりすると,
コマンドを送ることができなくなる.
そのような場合に,この機能を使用する.

RN4020は,
電源投入後5秒以内に\texttt{WAKE\_HW}ピンを3回以上フリップすることで,
工場出荷時の状態に戻る.
ジャンパーをRESETの設定にして{\tecS}に電源を投入すると,
{\tac}がこの操作を行う.

